\documentclass{article}
\usepackage{graphicx}
\usepackage{mhchem}
\usepackage{tikz}
\usepackage{pgfplots}
\usepackage{pgfplotstable}
\usepackage{filecontents}
\usepackage[T1]{fontenc}
\usepackage[utf8]{inputenc}
\usepackage[english, russian]{babel}

\title{ОПРЕДЕЛЕНИЕ СОДЕРЖАНИЯ ХЛОРИСТОВОДОРОДНОЙ КСЛОТЫ МЕТОДОМ КОНДУКТОМЕТРИЧЕСКОГО ТИТРОВАНИЯ}
\author{Афанасьев С.М.}
\date{06.05.24}

\usepackage{tocloft}
\renewcommand{\cftsecleader}{\cftdotfill{\cftdotsep}}
\renewcommand{\cftsecfont}{\normalfont}

\begin{document}
   \maketitle
   \tableofcontents
   \newpage
   
   \section*{Введение в кондуктометрию}
   \addcontentsline{toc}{section}{Введение в кондуктометрию}
    Кондуктометрия объединяет группу методов анализа, основанных на измерении электролитической проводимости
    исследуемых электролитов. Электролитической проводимостью называется способность вещества
    проводить электрический ток под действием внешнего электрического поля.
    \newline Кондуктометрия подразделяется на прямую и косвенную. Прямая кондуктометрия - это
    метод определениясодержания растворенного вещества путем непосредственного измерения электрической проводимости электролита
    известной химической природы.


   \section*{Введение в кондуктометрическое титрование}
   \addcontentsline{toc}{section}{Введение в кондуктометрическое титрование}
   \textit{Кондуктометрическое титрование} - метод анализа, основанный на определении эквивалентного объема титранта путем последовательного
   измерения электрической проводимости анализируемого раствора после добавления
   очередной порции взаимодействующего с ним титранта. В химическом анализе типов: нейтрализацияя,
   окисления-восстановления, осаждения и комплексообразования. Изменение электрической проводимости
   при титровании может быть различным, а значит, и характер кривых титрования в 
   зависимости от протекающей реакции тоже может быть различным. Однако все кривые кондуктометричекого титрования имеют излом
   в точке эквивалентности, которую находят по пересечению двух линейных участков на кривой титрования до и после нее.   
    
   Изменение удельной электропроводности (УЭП) $\chi$ в растворе соляной кислоты в процессе титрования 
   раствором гидроксида натрия до точки эквивалентности определяется высокой подвижностью Н$^+$-ионов, причем концентрация Н$^+$-ионов 
   в процессе титрования уменьшается пропорционально количеству добавленного титранта, и, соответственно, наблюдается линейное уменьшение 
   $\chi$ раствора. В точке эквивалентности удельная электропроводность приобретает минимальное значение. После точки эквивалентности 
   удельная электропроводность будет определяться подвижностью ОН$^-$-ионов, таким образом, $\chi$ линейно возрастает в соответствии 
   с увеличением добавленного избытка рабочего раствора гидроксида натрия.
    \newpage
    \section*{Экспериментальная проверка метода}
   \addcontentsline{toc}{section}{Экспериментальная проверка метода}
   \textbf{\textit{Цель работы:}} 
   закрепление навыков работы с аналитическим оборудованием кондуктометрического метода анализа. Проверка метода на относительную ошибку,
   учитывая то, что Erasttt делал этот метода на память (прочитав из книжки старой).
   \begin{flushleft}
    \textbf{\textit{Приборы и реактивы:}} \\
    \textit{кондуктометр;} \\
    \textit{стандартный раствор гидроксида натрия - 0,1 моль/л;} \\
    \textit{пластиковый стакан емкостью 100 мл;} \\
    \textit{бюретка ёмкостью 25 мл;} \\
    \textit{мерные пипетки емкостью 10 мл.} \\
    \textit{соляная кислота массой 0,01825 г.}
    \end{flushleft}
    
       \begin{filecontents*}{data_new.csv}
    $\chi$, V
    3.855, 0
    3.217, 1
    2.647, 2
    2.042, 3
    1.522, 4
    1.042, 5
    1.387, 6
    1.737, 7
    2.068, 8
    2.396, 9
    2.684, 10
    2.976, 11
    3.259, 12
    3.529, 13
    3.775, 14
\end{filecontents*}

\begin{figure}[h]
    \centering
    \begin{tikzpicture}
        \begin{axis}[
                scale only axis,
                width=13 cm,
                ylabel={$\chi$},
                xlabel={$V$},
                xmin=0, xmax=15,
                ymin=0, ymax=5,
                xtick={0,1,...,14},
                xticklabel style={anchor=north},
                ytick={0,1,...,4},
                axis lines=middle,
                grid=both
            ]
        \addplot[mark=*,mark options={color=blue}] table [col sep=comma, x=V, y=$\chi$] {data_new.csv};
        \draw[dashed] (axis cs:5,0) -- (axis cs:5,1.042);
    \end{axis}
    \end{tikzpicture}
    \caption{Кривая титрования контрольного раствора соляной кислоты}
\end{figure}
    Таким образом точка эквивалентности настала при объеме в 5 мл.\\
    Известную массу HCL найдем используя данное уравнение:
    \begin{equation}
        m(HCl) = C_{NaOH} \times V_{NaOH} \times \frac{M_{HCl}}{1000}
    \end{equation}
    \begin{equation}
        m(HCl) = 0.1 \times 5 \times \frac{36.5}{1000} = 0.01825 г.
    \end{equation}
    Относительная ошибка составляет:
     \begin{equation}
        W = \frac{0.01825-0.01825}{0.01825} = 0
    \end{equation}
    \textbf{Вывод:} Данный метод кондуктометрического титрования, который Эраст в тихую нашел и проверил. Полностью соответствует аналитическому анализу. Данный анализ прост в исполнении, не трубет длительного ожидания(как при потенциометрии).
    \end{document}
    
























 