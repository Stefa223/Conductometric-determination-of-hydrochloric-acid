\documentclass{article}
\usepackage{graphicx}
\usepackage{mhchem}
\usepackage{tikz}
\usepackage{pgfplots}
\usepackage{pgfplotstable}
\usepackage{filecontents}
\usepackage[T1]{fontenc}
\usepackage[utf8]{inputenc}
\usepackage[english, russian]{babel}

\title{ОПРЕДЕЛЕНИЕ СОДЕРЖАНИЯ ХЛОРИСТОВОДОРОДНОЙ КСЛОТЫ МЕТОДОМ КОНДУКТОМЕТРИЧЕСКОГО ТИТРОВАНИЯ}
\author{Афанасьев С.М.}
\date{06.05.24}

\usepackage{tocloft}
\renewcommand{\cftsecleader}{\cftdotfill{\cftdotsep}}
\renewcommand{\cftsecfont}{\normalfont}

\begin{document}
   \maketitle
   \tableofcontents
   \newpage

   \section*{Введение}
   \addcontentsline{toc}{section}{Введение}
  
   \textbf{\textit{Цель работы:}} 
   закрепление навыков работы с аналитическим оборудованием кондуктометрического метода анализа и методикой кондуктометрического титрования. 
   Закрепление полученных навыков и умений определять содержание соляной кислоты методом КТ.

   \begin{flushleft}
    \textbf{\textit{Приборы и реактивы:}} \\
    \textit{кондуктометр;} \\
    \textit{стандартный раствор гидроксида натрия - 0,1 моль/л;} \\
    \textit{пластиковый стакан емкостью 100 мл;} \\
    \textit{бюретка ёмкостью 25 мл;} \\
    \textit{мерные пипетки емкостью 10 мл.}
    \end{flushleft}
   \newpage

   \section*{Введение в кондуктометрическое титрование}
   \addcontentsline{toc}{section}{Введение в кондуктометрическое титрование}
   
   Изменение удельной электропроводности (УЭП) $\chi$ в растворе соляной кислоты в процессе титрования 
   раствором гидроксида натрия до точки эквивалентности определяется высокой подвижностью Н$^+$-ионов, причем концентрация Н$^+$-ионов 
   в процессе титрования уменьшается пропорционально количеству добавленного титранта, и, соответственно, наблюдается линейное уменьшение 
   $\chi$ раствора. В точке эквивалентности удельная электропроводность приобретает минимальное значение. После точки эквивалентности 
   удельная электропроводность будет определяться подвижностью ОН$^-$-ионов, таким образом, $\chi$ линейно возрастает в соответствии 
   с увеличением добавленного избытка рабочего раствора гидроксида натрия.

   \begin{figure}[h]
    \centering
    \begin{tikzpicture}
        \begin{axis}[
                scale only axis,
                width=7cm,
                ylabel=$\chi$,
                xlabel=$V$,
                xmin=0, xmax=16,
                ymin=0, ymax=5,
                xtick={0,1,...,14},
                xticklabel style={rotate=45, anchor=east},
                ytick={0,1.1,...,4},
                axis lines=middle,
                grid=both
            ]
            % Указывайте конкретные имена столбцов, замените xColumn и yColumn на соответствующие заголовки
            \addplot [only marks] table [x=$V$, y=$\chi$, col sep=comma] {data.csv};
        \end{axis}
    \end{tikzpicture}
    \caption{Кривая титрования контрольного раствора соляной кислоты}
\end{figure}

























   \end{document}
